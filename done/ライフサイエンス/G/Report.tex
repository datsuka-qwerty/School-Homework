\documentclass[a4j,uplatex]{jsarticle}
% グラフ関係
\usepackage[dvipdfmx]{graphicx}
\usepackage[dvipdfmx]{color}
% 数式関係
\usepackage{siunitx}
\usepackage{amsmath}
\usepackage{amssymb}
% 数式番号を(セクション番号.式番号)の形式にする e.g. (2.1)
\numberwithin{equation}{section}
% 表でHを使う
\usepackage{float}
% セルに斜線を入れる
\usepackage{diagbox}
% セル結合を使えるようにする
\usepackage{booktabs, multirow}
% 複数ページに渡る表を作る
\usepackage{longtable}
% 参考文献用
\usepackage{url}
% bulletより小さい丸,sbt
\newcommand{\sbt}{\,\begin{picture}(-1,1)(-1,-3)\circle*{3}\end{picture}\ }

\title{誰もが運動・スポーツに参画できる環境や仕組みについて}
\author{5年L組23番 塚田蓮大}
\date{\today}

\begin{document}
	\maketitle

	\section{オンラインで運動・スポーツの疑似体験}
		インターネットを用いて、実際にその場で運動しているかのような疑似体験を指せるデバイスを、高専で学んだプログラミング技術やロボット技術を活かし実現したい。
		具体的には、既に存在するVRやAR, MRなどの技術を駆使して視覚的な情報の他に実際に山の地形を再現するようなランニングマシンのようなデバイスを用いることで、現地で運動しているかのような体験を、
		時間的制限や、地理的制限を受けている人々でも気軽に行えるようにしたい。
	\section{運動・スポーツに関するAIトレーナー}
		運動・スポーツを継続的に行えるようにAIを用いた運動トレーナーを作成することで、個人個人に合ったトレーニングメニューや、その人が興味を持っているであろう運動・スポーツ・健康に関する情報を提供できるような
		AIトレーナー・AIコンシェルジュを、高専で学んだプログラミング技術、デザイン技術、統計情報知識を用いて作成し、多くの人々が継続的に運動を行えるようにしたい。
	\section{運動・スポーツの成果を共有できるサービス}
		自身が行った運動・スポーツ・トレーニングの成果を友人や家族、一般の人々に共有できるようなサービスを高専で学んだネットワーク技術、デザイン技術、プログラミング技術を用いて作成することで、
		運動・スポーツに対する意欲を増加させ、周りの人がやっているなら自分も始めてみようという気持ちを起こさせ、より多くの人々に運動・スポーツを気軽に行ってもらえる環境を作りたい。
	\section{運動していると錯覚させ、イメージトレーニングを行えるデバイス}
		脳の神経ネットワークを解析・利用できるまで研究を行い、脳内にある特定のニューロンを任意のタイミングで刺激できるようなデバイスを直接接続し制御することで、
		体を一切動かしていなくても、全くの未経験でも、既に経験したことのある人のニューロンの動きを再現することでイメージトレーニングや、ハンディキャップを抱えているような人へスポーツの体験をできるようにしたい。
\end{document}
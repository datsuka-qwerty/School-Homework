\documentclass[a4j,uplatex]{jsarticle}
% グラフ関係
\usepackage[dvipdfmx]{graphicx}
\usepackage[dvipdfmx]{color}
% 画像の余白を調整
\setlength\intextsep{0pt}
\setlength\textfloatsep{0pt}
% 数式関係
\usepackage{siunitx}
\usepackage{amsmath}
\usepackage{amssymb}
% 数式番号を(セクション番号.式番号)の形式にする e.g. (2.1)
\numberwithin{equation}{section}
% 表でHを使う
\usepackage{float}
% セルに斜線を入れる
\usepackage{diagbox}
% セル結合を使えるようにする
\usepackage{booktabs, multirow}
% 複数ページに渡る表を作る
\usepackage{longtable}
% 参考文献用
\usepackage{url}
% bulletより小さい丸,sbt
\newcommand{\sbt}{\,\begin{picture}(-1,1)(-1,-3)\circle*{3}\end{picture}\ }

\title{\vspace{-5cm}ネットワークアーキテクチャ12週目課題 システムアーキテクチャについて}
\author{5年L組23番 塚田蓮大}
\date{\today}

\begin{document}
\maketitle

\section*{システムアーキテクチャとは}
\begin{itemize}
	\item OSについて \mbox{}\\
	      OS(Operating System)とは、ハードウェアと密接に関わりながら、利用しやすいシステム環境を構築するソフトウェアである。(e.g. Windows, Mac OS, Linux)
	\item OSの目的 \mbox{}\\
	      \begin{itemize}
		      \item 開発者からみたOS \mbox{}\\
		            様々なアプリケーションで共通する機能は、アプリケーション毎に用意するのではなく、OSが提供して欲しい。
		            ->必要な機能をAPI(Application Programming Interface)として提供する。
		      \item ユーザーからみたOS \mbox{}\\
		            様々なアプリケーションでも操作感は同じであって欲しい。
		            ->ユーザーの操作環境をOSが提供し、操作をある程度共通化する。また、GUIにより直感的でわかりやすい操作を実現する。
	      \end{itemize}
	\item プロセス管理 \mbox{}\\
	      \begin{itemize}
		      \item プロセスの種類 \mbox{}\\
		            \begin{itemize}
			            \item OSプロセス \mbox{}\\
			                  OSから発生した特権モードで実行されれるプロセス\\
			                  ユーザモードで実⾏されるプログラムからの割込みにより起動する
			            \item ユーザープロセス \mbox{}\\
			                  ユーザモードで実⾏されるプロセス\\
			                  OSによって起動される
		            \end{itemize}
		      \item プロセススケジューラ \mbox{}\\
		            OSが持つプロセス管理機能であり、プロセスの状態管理、• 実⾏するプロセスの選択と選択したプロセスへのCPUの割り当てを行う。
	      \end{itemize}
\end{itemize}
\section*{初めて使ったOS}
初めて使ったOSはWidnows98である。
\end{document}
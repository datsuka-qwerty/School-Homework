\documentclass[a4j,uplatex]{jsarticle}
% グラフ関係
\usepackage[dvipdfmx]{graphicx}
\usepackage[dvipdfmx]{color}
% 数式関係
\usepackage{siunitx}
\usepackage{amsmath}
\usepackage{amssymb}
% 数式番号を(セクション番号.式番号)の形式にする e.g. (2.1)
\numberwithin{equation}{section}
% 表でHを使う
\usepackage{float}
% セルに斜線を入れる
\usepackage{diagbox}
% セル結合を使えるようにする
\usepackage{booktabs, multirow}
% 複数ページに渡る表を作る
\usepackage{longtable}
% 参考文献用
\usepackage{url}
% bulletより小さい丸,sbt
\newcommand{\sbt}{\,\begin{picture}(-1,1)(-1,-3)\circle*{3}\end{picture}\ }

\title{コンピュータアーキテクチャ2週目課題}
\author{5年L組23番 塚田蓮大}
\date{\today}

\begin{document}
	\maketitle

	\section{ノイマン型コンピュータの基本}
		ノイマン型コンピュータには大きく3つの特徴がある。
		\begin{enumerate}
			\item プログラム記憶方式 \mbox{}\\
				プログラムを内部のメモリに格納することで、プログラムの入力・変更を簡単に
			\item 逐次処理方式  \mbox{}\\
				原則、命令は実行順にメモリに格納する。順次取り出し、処理を実行する。\\
				取り出す命令のアドレスは、プログラムカウンタによって指示される。
			\item 単一メモリ方式 \mbox{}\\
				プログラムとデータは、同じメモリに格納。メモリにはアドレスが割り当てられる。 \\
				一時的なデータ格納領域として、レジスタを備える。
		\end{enumerate}
	\section{ノイマン型コンピュータの基本構成}
		ノイマン型コンピュータには5つの基本的な役割から成る。
		\begin{enumerate}
			\item 演算装置 \mbox{}\\
				算術演算・論理演算を行う。
			\item 制御装置 \mbox{}\\
				全ての装置の制御する装置。
			\item 記憶装置 \mbox{}\\
				データやプログラムを記憶する装置。一般的にRAMと呼ばれる主記憶装置と、SSD・HDDなどが使われる補助記憶装置から成る。
			\item 入力装置 \mbox{}\\
				プログラムやデータを主記憶装置に入力するための装置。
			\item 出力装置 \mbox{}\\
				処理された結果を出力する装置
		\end{enumerate}
		また、演算装置・制御装置を2つ合わせて中央処理装置(CPU:Central Processing Unit)と呼称する。
	% \section{ノイマン型コンピュータの基本動作}
	% 	\begin{enumerate}
	% 		\item PCに格納されたアドレスをMAR(メモリアドレスレジスタ)に送る。
	% 		\item 指定アドレスの命令をIR(命令ジスタ)に取り出す。
	% 		\item IRの命令をDEC(デコーダ)に送り復号。
	% 		\item 命令の実行に必要な制御信号を出力。
	% 		\item 制御信号に従い演算等の処理を実行。
	% 		\item 次の命令が格納されたアドレスをPCにセットする。
	% 	\end{enumerate}
\end{document}
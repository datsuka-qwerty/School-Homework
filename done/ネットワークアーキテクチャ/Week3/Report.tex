\documentclass[a4j,uplatex]{jsarticle}
% 段落初めの余白を削除
% \setlength\parindent{0pt}
% 行間のポイント数調整
\renewcommand{\baselinestretch}{0.8}
% グラフ関係
\usepackage[dvipdfmx]{graphicx}
\usepackage[dvipdfmx]{color}
% 数式関係
\usepackage{siunitx}
\usepackage{amsmath}
\usepackage{amssymb}
% 数式番号を(セクション番号.式番号)の形式にする e.g. (2.1)
\numberwithin{equation}{section}
% 表でHを使う
\usepackage{float}
% セルに斜線を入れる
\usepackage{diagbox}
% セル結合を使えるようにする
\usepackage{booktabs, multirow}
% 複数ページに渡る表を作る
\usepackage{longtable}
% 参考文献用
\usepackage{url}
% bulletより小さい丸,sbt
\newcommand{\sbt}{\,\begin{picture}(-1,1)(-1,-3)\circle*{3}\end{picture}\ }

\title{\vspace{-5cm}ネットワークアーキテクチャ3週目課題命令セットについて}
\author{5年L組23番 塚田蓮大}
\date{\today}

\begin{document}
	\maketitle

	命令セットとはCPUが備えている命令の一覧を指す。使いたいCPUがどのような操作を行えるのかを調べる際に用いる。\\
	以下は情報処理技術者試験用に想定されたコンピュータ向けのアセンブラ言語(CASL Ⅱ)の命令セットである。
	\begin{enumerate}
		\item ロード、ストア、ロードアドレス命令 \mbox{}\\
			・レジスタやメモリ間などのデータ移動を行う命令
		\item 算術・論理演算命令
		\item 比較演算命令 \mbox{}\\
			・2個のデータの大小を比較し、その結果をフラグレジスタFRに反映する命令
		\item 分岐命令 \mbox{}\\
			・条件分岐命令(FRの状態に応じた分岐)と無条件分岐命令(FRの状態によらない分岐)
		\item スタック操作命令 \mbox{}\\
			・PUSHとPOPの命令
		\item コール、リターン命令 \mbox{}\\
			・サブルーチンの呼び出し、復帰命令
		\item その他
	\end{enumerate}
	\begin{table}[htbp]
		\begin{minipage}[c]{0.5\hsize}
			\centering
			\caption{ロード、ストア、ロードアドレス命令}
			\begin{tabular}{|c|c|}
				\hline ロード LoaD & LD \\ \hline
				ストア STore & ST \\ \hline
				ロードアドレス LoadADdress & LAD \\ \hline
			\end{tabular}
		\end{minipage}
		\begin{minipage}[c]{0.5\hsize}
			\centering
			\caption{算術、論理演算命令}
			\begin{tabular}{|c|c|}
				\hline 算術加算 ADD Artihmetic & ADDA \\ \hline
				論理加算 ADD Logical & ADDL \\ \hline
				算術減算 SUBtract Arithmetic & SUBA \\ \hline
				論理減算 SUB Logical & SUBL \\ \hline
				論理積 AND & AND \\ \hline
				論理和 OR & OR \\ \hline
				排他的論理和 eXclusive OR & XOR \\ \hline
			\end{tabular}
		\end{minipage}
		\begin{minipage}[c]{0.5\hsize}
			\centering
			\caption{比較演算命令}
			\begin{tabular}{|c|c|}
				\hline 算術比較 ComPare Arithemetic & CPA \\ \hline
				論理費 ComPare Logical & CPL \\ \hline
			\end{tabular}
		\end{minipage}
		\begin{minipage}[c]{0.5\hsize}
			\centering
			\caption{分岐命令}
			\begin{tabular}{|c|c|}
				\hline 正分岐 Jump on PLus & JPL \\ \hline
				負分岐 Jmp on MInus & JMI \\ \hline
				非零分岐 Jump on Non Zero & JNZ \\ \hline
				零分岐 Jump on ZEro & JZE \\ \hline
				オーバーフロー分岐 Jump on OVerflow & JOV \\ \hline
				無条件分岐 unconditional JUMP & JUMP \\ \hline
			\end{tabular}
		\end{minipage}
	\end{table}
	\begin{table}[htbp]
		\begin{minipage}[c]{0.5\hsize}
			\centering
			\caption{スタック操作命令}
			\begin{tabular}{|c|c|}
				\hline プッシュ PUSH & PUSH \\ \hline
				ポップ POP & POP \\ \hline
			\end{tabular}
		\end{minipage}
		\begin{minipage}[c]{0.5\hsize}
			\centering
			\caption{コール、リターン命令}
			\begin{tabular}{|c|c|}
				\hline コール CALL subroutine & CALL \\ \hline
				リターン RETurn from subroutine & RET \\ \hline
			\end{tabular}
		\end{minipage}
		\begin{minipage}[c]{0.5\hsize}
			\centering
			\caption{その他}
			\begin{tabular}{|c|c|}
				\hline スーパーバイザコール SuperVisor Call & SVC \\ \hline
				ノーオペレーション No OPeration & NOP \\ \hline
			\end{tabular}
		\end{minipage}
	\end{table}
	\begin{table}
		
	\end{table}
\end{document}
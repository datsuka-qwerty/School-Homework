\documentclass[a4j,uplatex]{jsarticle}
% グラフ関係
\usepackage[dvipdfmx]{graphicx}
\usepackage[dvipdfmx]{color}
% 数式関係
\usepackage{siunitx}
\usepackage{amsmath}
\usepackage{amssymb}
% 数式番号を(セクション番号.式番号)の形式にする e.g. (2.1)
\numberwithin{equation}{section}
% 表でHを使う
\usepackage{float}
% セルに斜線を入れる
\usepackage{diagbox}
% セル結合を使えるようにする
\usepackage{booktabs, multirow}
% 複数ページに渡る表を作る
\usepackage{longtable}
% 参考文献用
\usepackage{url}
% bulletより小さい丸,sbt
\newcommand{\sbt}{\,\begin{picture}(-1,1)(-1,-3)\circle*{3}\end{picture}\ }

\title{ネットワークアーキテクチャの概要}
\author{5年L組23番 塚田蓮大}
\date{\today}

\begin{document}
	\maketitle
	\section{コンピュータアーキテクチャとは}
		アーキテクチャ(Architecture)とは構造や構成というような意味を持ち、
		特にコンピュータアーキテクチャはハードウェアとソフトウェア、さらにはコンピュータの設計思想や開発技術を包括した広い意味を持つ
		\begin{itemize}
			\item 命令セットアーキテクチャ
			\item 演算アーキテクチャ
			\item 制御アーキテクチャ
			\item I/Oアーキテクチャ
			\item ネットワークアーキテクチャ
			\item システムアーキテクチャ
		\end{itemize}
		これらすべてまとめてコンピュータアーキテクチャと呼称する。

	\section{コンピューターの歴史}
		\subsection{機械式計算機}
			歯車等を用いて計算を行えるように\\
			後に、パンチカードを用いてプログラムによって自動的に計算が行える様になった。
		\subsection{電子式計算機}
			真空管を用いた第1世代、トランジスタを用いた第2世代、ICを用いた第3世代とあり、ノイマン型コンピュータ(現代のコンピュータの原型)が出来上がった。
	\section{コンピュータの分類}
		\begin{itemize}
			\item マイクロコンピュータ
			\item パーソナルコンピュータ
			\item タブレット
			\item ワークステーション
		\end{itemize}
		
\end{document}
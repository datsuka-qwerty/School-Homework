\documentclass[a4j, twocolumn, uplatex]{jsarticle}
% グラフ関係
\usepackage[dvipdfmx]{graphicx}
\usepackage[dvipdfmx]{color}
% 数式関係
\usepackage{siunitx}
\usepackage{amsmath}
\usepackage{amssymb}
% 数式番号を(セクション番号.式番号)の形式にする e.g. (2.1)
\numberwithin{equation}{section}
% 表でHを使う
\usepackage{float}
% セルに斜線を入れる
\usepackage{diagbox}
% セル結合を使えるようにする
\usepackage{booktabs, multirow}
% 複数ページに渡る表を作る
\usepackage{longtable}
% 参考文献用
\usepackage{url}
% bulletより小さい丸,sbt
\newcommand{\sbt}{\,\begin{picture}(-1,1)(-1,-3)\circle*{3}\end{picture}\ }

\title{\vspace{-5cm}情報システム工学 レポート1}
\author{自動車業界の半導体供給不足問題 \\ \today Lクラス23番 塚田蓮大}
\date{}

\begin{document}
	\maketitle
	筆者は、自動車業界の半導体供給不足問題が単なる一過性のものではないと主張している。
	その理由として、自動車アプリケーション用の製造設備、自動車アプリケーションの要求品質、自動車用半導体部品とゲーム機やサーバーなどのコンピュータ機器用半導体部品の価格差の
	3つの点がある。
	\section{自動車アプリケーション用の製造設備}
		自動車アプリケーション用の半導体は、28nm以下のプロセスで製造されている。一方、半導体業界の最先端は10nm以下のプロセスであり、今後もさらに微細化が進むと予想される。そのため、自動車アプリケーション用の半導体は、半導体業界全体の技術トレンドから見ると、すでに成熟した技術と言える。
		半導体製造は、設備投資に莫大な費用がかかる。そのため、半導体ベンダーは、将来性のある技術に投資を行いたいと考えるのが一般的である。しかし、自動車アプリケーション用の半導体は、技術的に成熟しているため、半導体ベンダーは増強投資に魅力を感じにくい。
		このため、自動車アプリケーション用の半導体製造設備は、老朽化や生産能力不足に陥っているケースも多い。そのため、供給不足が起こりやすい状況にある。

	\section{自動車アプリケーションの要求品質}
		自動車アプリケーションの顧客の要求スペックは、品質、カスタム性、長期サポート、コストなどにおいて非常に厳しい。そのため、半導体ベンダーにとっては割に合わないビジネスである場合が多い。
		具体的には、自動車アプリケーション用の半導体は、自動車の安全性や信頼性に直結するため、非常に高い品質が求められる。また、自動車メーカーは、独自の設計や機能の搭載を求めるケースも多いため、半導体ベンダーはカスタマイズに対応する必要がある。さらに、自動車は長期間使用されるため、長期的なサポートも求められる。
		このような顧客の要求を満たすためには、半導体ベンダーは、コストをかけて技術開発や製造設備の投資を行う必要がある。しかし、半導体ベンダーにとっては、これらのコストが回収できるだけの利益が見込めない場合も多い。
		このため、半導体ベンダーは、自動車アプリケーションへの投資を積極的に行うよりも、利益率の高い他のアプリケーションに投資を行う傾向にある。

	\section{自動車用半導体部品とゲーム機やサーバーなどのコンピュータ機器用半導体部品の価格差}
		自動車用半導体部品は、ゲーム機やサーバーなどのコンピュータ機器用半導体部品と比べて、価格が安い。
		これは、自動車アプリケーションの顧客である自動車メーカーが、コストを重視する傾向にあるためである。
		このため、半導体ベンダーは、利益率の高いコンピュータ機器用半導体部品に注力し、自動車用半導体部品への投資を抑制する傾向にある。

	\section{まとめ}
		以上の観点から、自動車業界の半導体供給不足問題は、単なる一過性のものではないと言える。
		自動車アプリケーション用の製造設備の老朽化や生産能力不足、自動車アプリケーションの要求品質の厳しさ、自動車用半導体部品とコンピュータ機器用半導体部品の価格差などの要因が、半導体供給不足を長期化する可能性がある。
		この問題を解決するためには、自動車メーカーや半導体ベンダーが、自動車アプリケーションへの投資を増やし、自動車用半導体部品の価格を高めるなどの取り組みが必要である。

\begin{thebibliography}{99}
		\bibitem{theonlyone}吉川明日論の半導体放談 第 173 回, "		自動車用アプリケーションが本流とならない半導体ベンダー側の事情", \url{https://news.mynavi.jp/techplus/article/semicon-173/}, アクセス日{\today}
	\end{thebibliography}
\end{document}